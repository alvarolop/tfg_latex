\documentclass[12pt,a4paper,english,twoside]{book} %book, report, article
\usepackage{babel} %%%Incluimos el paquete Babel que sirve para separar correctamente las palabras de multitud de idiomas%%%
\usepackage[utf8]{inputenc}%%%Este paquete permite poner acentos directamente%%%
\usepackage{amsmath}%%%Macros AMS%%%
%\usepackage{amsthm}%%%Macros AMS para teoremas%%%
%\usepackage{amsfonts}%%%Permite usar fuentes AMS%%%
\usepackage[dvips]{epsfig} %%%Inclusión de figuras postscript con visualización posterior%%%
\usepackage{indentfirst} %%%Espaciado de primera línea de cada párrafo%%%
\usepackage[gen]{eurosym} %soporte para el comando \euro
\usepackage{hyperref}
\usepackage{graphicx}
\graphicspath{{images/}}
\usepackage{anysize}
\usepackage{gensymb}
\marginsize{4cm}{3cm}{3cm}{2cm} %define márgenes {izdo}{drch}{sup}{inf}
\usepackage{rotating}
\usepackage{ulem}
\usepackage{makeidx}
%\usepackage{showidx}
\usepackage{color}
\usepackage{listings}

\usepackage{color}
\usepackage{listings}
\usepackage{titlesec}
\usepackage[final]{pdfpages} %Permite insertar pdf's enteros o por paginas

\usepackage{hyperref}
\usepackage{boxedminipage}
\usepackage{verbatim}
\usepackage{array}
\usepackage{multirow}
\usepackage{subfigure}

%Paquetes de Alvaro
\usepackage{enumerate} % enumerados
%\begin{enumerate}[A.]
%    \item Manzanas.
%    \item Plátanos.
%    \item Pescado fresco.
%\end{enumerate}
\usepackage{float} % para usar [H]
\usepackage{cite}
\usepackage{url}
\usepackage{wrapfig}


% \newcommand{\metodo}[1]{\ttfamily{#1}\normalfont }
% \definecolor{listinggray}{gray}{0.9}
% \newcommand{\codigo}[3]{\lstset{language=c}\lstset{backgroundcolor=\color{listinggray},rulecolor=\color{black}}\lstset{commentstyle=\textit,showspaces=false,tabsize=4}\lstset{linewidth=\textwidth}\lstset{frame=TrBl,framerule=1.5pt}\lstinputlisting[caption=#1,label=#2]{#3}}, stringstyle=\upshape
% \newcommand{\cabecera}[1]{\ttfamily{#1}\normalfont}

\newenvironment{codigo}[1]{\flushleft\boxedminipage{\linewidth}\fontsize{#1}{#1}\selectfont\verbatim}{\endverbatim\endboxedminipage\endflushleft}

\pretolerance=10000 %para que no separe las palabras al cambiar de linea
\tolerance=10000

% Personalización
\titleformat{\chapter}[display] %Define formato de los capítulos
{\Huge\filright}			%format	
{\Huge\scshape{\chaptertitlename}	%'chapter number' format
 \Huge\thechapter
}
{5mm}{\bfseries}

\makeindex

\begin{document}
\addtolength{\oddsidemargin}{-1cm}
\addtolength{\evensidemargin}{-1cm}
\pagestyle{empty}


\makeatletter
\def\@department{Departamento de Ingeniería Electrónica}
\def\@university{Universidad Politécnica de Madrid}
\def\@school{Escuela Técnica Superior \\de Ingenieros de Telecomunicación}

\def\@author{Álvaro López Medina}
\def\@tutor{Dña. Marina Zapater Sancho}
\def\@advisor{D. Jose Manuel Moya Fernández}

\def\@presidente{D. Rubén San Segundo Hernández}
\def\@vocal{D. José Manuel Moya Fernández}
\def\@secretario{D. Juan Mariano de Goyeneche y Vázquez de Seyas}
\def\@suplente{D. Juan Manuel Montero Martínez}

\def\@logo{logoETSIT}
\def\@title{Analysis and design of multi-level actuation policies to minimize the energy consumption of enterprise servers}
\def\@degree{Trabajo de Fin de Grado}
\def\@date{July 2015} 



\begin{titlepage}

% Upper space
%   \vspace*{2cm}

% Affiliation
{\centering\scshape\setlength{\parindent}{0cm} \LARGE \@university\\[20mm]}
{\centering\scshape\setlength{\parindent}{0cm} \Large \@school\\[10mm]}

\begin{center}

% Logo
\includegraphics[width=0.7\textwidth]{logoETSIT}


% Affiliation - logo space
\vspace{1cm}

% Type of document: PhD Thesis...
{\Huge \centering \scshape \@degree\\[1cm]}

% At least... the title
{\Huge \bfseries \@title \par}

% Title - logo space
 \vspace{3cm}

% Title - name space
% \vspace{3cm}

% Author's name
{\large\sc \@author \\[1cm]}

% Date
{\large\sc \@date}\\

\end{center}
\end{titlepage}



%%% Title page definition
\begin{titlepage}

\vspace{2cm}

\large
{\centering\LARGE\bfseries Proyecto Fin de Carrera\\[1cm]}
\begin{tabular}[t]{p{3.5cm} l}
 \bf Título: & \sc Analysis and design of multi-level actuation\\
             & \sc  policies to minimize the energy consumption\\
             & \sc  of enterprise servers. \\
& \\ 
\bf Autor:  & \sc D. \@author\\ \\
\bf Tutor: & \sc \@tutor \\ \\
\bf Ponente: & \sc \@advisor \\ \\
\bf Departamento: & \sc \@department \\
\end{tabular}
\bigskip
\vspace{1cm}

\begin{center}
\LARGE\bfseries Miembros del tribunal
\end{center}

\vspace{1cm}
\large
\begin{tabular}[t]{p{3.5cm} p{11cm}}
\bf Presidente: & \sc \@presidente \\ \\
\bf Vocal: & \sc  \@vocal \\ \\
\bf Secretario: & \sc \@secretario \\ \\
\bf Suplente: & \sc \@suplente \\ \\
\end{tabular}

\vspace{1cm}

\vspace{2cm}
\begin{flushright}
 Madrid, a \hspace{0.5cm} de \hspace{3cm} de 

\end{flushright}
\end{titlepage}





\begin{titlepage}

% Upper space
%   \vspace*{2cm}

% Affiliation
{\centering\scshape\setlength{\parindent}{0cm} \LARGE \@university\\[20mm]}
{\centering\scshape\setlength{\parindent}{0cm} \Large \@school\\[10mm]}

\begin{center}

% Logo
\includegraphics[width=0.7\textwidth]{logoETSIT}


% Affiliation - logo space
\vspace{1cm}

% Type of document: PhD Thesis...
{\Huge \centering \scshape \@degree\\[1cm]}

% At least... the title
{\Huge \bfseries \@title \par}

% Title - logo space
 \vspace{3cm}

% Title - name space
% \vspace{3cm}

% Author's name
{\large\sc \@author \\[1cm]}

% Date
{\large\sc \@date}\\
\end{center}

\end{titlepage}



\makeatother


\frontmatter
\pagestyle{headings}
\addtolength{\parskip}{+.25cm}
\addtolength{\parindent}{+.5cm}
\addtolength{\oddsidemargin}{+1cm}
\addtolength{\evensidemargin}{+1cm}

%%%%%%%%%%%%%%%%%%%%%%%%%%%%%%%%%%%%%%%%%%%%%%%%%%%%%%%

\include{Abstract}
\chapter{List of acronyms}
%\addcontentsline{toc}{chapter}{Lista de Acrónimos}
\begin{tabular}{ p{3cm} p{10cm}}
  \bf DC &  Data Centers \\ \\
  \bf OCP &  Open Compute Project \\ \\
  \bf PUE & Power Usage Effectiveness \\ \\
  \bf DIMM & Dual In-line Memory Module  \\ \\
  \bf RPM &  Revolutions Per Minute \\ \\
  \bf RAID & Redundant Array of Independent Disks  \\ \\
  \bf IPMI & Intelligent Platform Management Interface  \\ \\
  \bf BMC &  Baseboard Management Controller \\ \\
  \bf BIOS & Basic Input/Output System \\ \\
  \bf OS   & Operating System \\ \\
  \bf GRUB & GRand Unified Bootloader  \\ \\
  \bf SPEC & Standard Performance Evaluation Corporation  \\ \\
  \bf EDP &  Energy Delay Product \\ \\
  \bf BMC &  Baseboard Management Controller \\ \\
  \bf FSC &  Fan Speed Control \\ \\
  \bf PWM &  Pulse Width Modulation  \\ \\
  \bf CFM &  Cubic Feet per Minute \\ \\
  \bf PID &  Proportional-Integral-Derivative \\ \\
  \bf IPC &  Instructions per cycle  \\ \\

\end{tabular}



\tableofcontents
\listoffigures
\listoftables

%%%%%%%%%%%%%%%%%%%%%%%%%%%%%%%%%%%%%%%%%%%%%%%%%%%%%%
\mainmatter
%\marginsize{6cm}{2.5cm}{3cm}{2cm}

%%%%%%%%%%%%%%%%%%%%%%%%%%%%%%%%%%%
% Nota: Para mayor manejabilidad, se puede incluir cada capítulo en un archivo
% separado, con el comando "input"
%\input{introduccion}

\chapter{Introduction and Objectives}
\section{Background and motivation}

Today, due to the need of minimizing costs, power saving is one of the key aspects in almost every technological project. Those companies whose activities require more power consumption are more interested in develop energy aware policies in order to save costs. 

Moreover, nowadays, cloud computing has revolutionized the information technology (IT) industry by enabling elastic on-demand provisioning of computer resources. Cloud computing has resulted in the establishment of big Data Centers (DC) containing the infrastructure needed. Nevertheless, this server concentration makes data centers one of the greatest users of energy around the world.

According to Koomey \cite{koomey2011growth}, the electricity used in 2010 by world data centers represented between 1.1 and 1.5\% of the total amount, whereas in the US data centers, the electricity represented between 1.7 and 2.2\%. Nowadays, the contribution of data centers to European electricity consumption is estimated to be around 2 to 2.5\%. \cite{datoEUROPAConsumo}

The paragraph above shows a clear growing trend. This trend is caused by two factors: the continuous growing of the number of DCs and racks, and the increment in power demand per rack. This is owing to the existence of more efficient servers, but also more powerful, so they need more power to run. Minimizing power consumption of DC will save a large amount of energy.

Given this power consumption data, it is important to study which are the main contributors in a normal DC. First of all, we will divide them in two groups: on the one hand, the power consumption of the servers, which is called "IT Energy" and on the other hand the consumption of the remaining systems that cover the needs of servers - cooling, lightning, etc. - plus the energy losses due to inefficiencies in the power distribution network. There are several lines of work that explore minimization of the expenditure for both contributions. \cite{Brady2013}

Nowadays, there are various lines on how to save energy in Data Centers. For example, the Open Compute Project \cite{ocpHomepage} provides a set of specifications to be met by green servers. There are also some server-control policies defined to save energy. These policies are usually defined and implemented at low level - i.e. firmware - and are transparent for the user. Some processors and fans, have power-saving state that allow them to save energy.

The research of the GreenLSI team \cite{grenlsiHP} is focused on the development of holistic optimization policies  to minimize energy in data centers. Their research results have provided some complex policies that can be applied at various abstraction levels, from the server to the data center. However, some of their proposed policies currently lack the actuation support needed to be implemented in a non-intrusive ways in commercial servers. In this sense, the goal of this bachelor thesis is to fill this gap, analyzing how actuation techniques can be implemented at the highest level possible, providing actuation support to the optimizations developed by researchers.

The IT Equipment will be analyzed in order to create actuation support that could minimize this power consumption. Firstly, analyzing which components of the server can be managed and at which level of abstraction. Secondly, studying the consumption behavior depending on the power configuration selected. Finally, with this information, some conclusions can be drawn.

\section{Objectives}
Therefore, the aim of this thesis  is to analyze control techniques to reduce the consumption of DCs. In order to perform this task, each component of the server will be examined separately and new control techniques will be implemented to analyze the consumption minimization.
Accordingly, the main objectives for the present work are:
\begin{itemize}
    \item Analysis of the main subsystems of the server that have an impact on power in order to understand the power saving that can be achieved in each system.
    \item Analysis of several levels of abstraction so as to find the highest level in which each of the subsystems previously mentioned can be implemented.
    \item Analysis of the impact of each actuation support to find out which reduces more the consumption.
    \item Development of the most suitable multi-level actuation techniques to minimize the energy consumption.
    \item Analysis of actuation policies to optimize energy.
\end{itemize}


\chapter{Development and Results}

%Once the state of the art has been studied and the main lines of work of energy reduction are known, analysis of all the techniques will be made.
\input{Chapter21}
\clearpage
\input{Chapter22}
\input{Chapter23}
%%\input{Chapter33}
\input{Chapter24}
\clearpage

\input{Chapter25}
\clearpage

\input{Chapter26}
\chapter{Conclusions}

Nowadays, reducing the power consumption and therefore costs is one of the main goals of any business. Specially in data centers, where the largest item of expenditure is the power cost, it is a critical goal.

In this thesis, the main objective has been analyzing all non-intrusive actuation control techniques for enterprise servers, focusing in the OCP Decathlete in order to provide support to the optimization of researches. This server has been chosen due to it is open design and all the customization possibilities this involves.

First, it has been analyzed how to change the configuration of memories, processors and fans in several ways. Then, the second phase had the aim of managing them at user space level in the OS. This would generalise this new technique for as many server models as possible helping the GreenLSI to implement it in all their servers.

For those servers that can be currently controlled, some benchmarks have been tested in order to model the behaviour of the server using different power configurations. Thanks to the results of the benchmarks, the relation between the power consumption and the performance has been quantified.

Finally, it has been studied the implications of using DVFS or turning off some CPUs in each type of workload in order to define a new saving policy to apply in the server.

\newpage
\section{Future Work}

Here I will list the main points I think future works should be focused in order to achieve the goal of controlling this enterprise server.

\emph{Future actions on actuation techniques}

\begin{itemize}

\item Firstly, it is important to rise the actuation level of some subsystems, specially fans control which can only be managed using the SP, to the OS level. This effort will help the GreenLSI to control the fan speed as they want to be able to create solid actuation policies.

\item During this project we analyzed the firmware deployed at the SP and we believe it would be possible to add support at the firmware level for fan control.



\emph{Research items}

\item At the end of this thesis we have provided a way to enable and disable memory DIMMs. However, a deep assessment on the power/energy/EDP benefits of this approach needs to be investigated. Moreover, this open new research challenges from the workload allocation perspective, allowing to decide the amounts of DIMMs active for each workload.

\item Finally, now that it is known how it affects to the server the DVFS and the CPU's off, it is necessary to create a complex policy based on the different type of workloads - CPU intensive and Memory intensive - in order to save the maximum amount of energy as possible combining those techniques.
\end{itemize}


\cleardoublepage
 \backmatter
\nocite{*}
\bibliographystyle{unsrt} %unsrt %plainurl %ieeetr
\bibliography{References}
\addcontentsline{toc}{part}{References}

%\input{References}

\printindex

\cleardoublepage
\newpage
\vspace*{13cm}

%\hline

\begin{center}
This page is left blank intentionally.
\end{center}
\end{document}